\chapter{What is the innovation language paradox and how to deal with it?}
\label{c3}
\chaptermark{What is the innovation language paradox and how to deal with it?}

\abstract{Analysis and design of an innovation language revolves around continuously improving the clarity, precision, and accuracy of definitions that constitute that language. At the same time, I argued that we should want to speed innovation up. I argue in this chapter that this leads to a paradox, and that the resolution of the paradox makes us progress in the design of an innovation language.}


\section{What is the innovation language paradox?}
\label{c3:s1}
In chapter \ref{c2}, I outlined an approach to the design of innovation languages. That approach should look odd, because definitions are typically associated with knowledge, that is, statements which usually need to satisfy many hard conditions. There has to be evidence of truth, which may amount to evidence for fitness between the statement and that which it is about when it is observed; and there need to be reproducible procedures to generate such evidence. Look at dictionaries, encyclopaedia, academic, and technical literature. We make definitions for what we know enough to want to pin it down, and for most practical purposes, to keep it that way until there are strong reasons for change.

You are served many definitions during formal education, when you get them precisely because they are, and have been considered as useful by a community that worked on the underlying knowledge for some time. Those definitions graduated scrutiny, or better, they survived it. By learning them, you prepare yourself for situations when you might need to coordinate, to work with others who learned them too, and this should help you align on what needs to be done. Definitions, in short, are one of the few enduring ways with which we have been trying to move knowledge around.

Where's the paradox?

The approach from chapter \ref{c2} suggests producing definitions for terms that play a role in new ideas. At the same time, we want to be precise, accurate, and clear in those definitions. And we want to progress through these ideas, essentially replacing them with more convincing ones. Finally, we want to do this quickly. 

In other words, the approach says that we should make definitions of ideas we want to quickly replace. 

Isn't this a paradox? Making a definition takes time, yet we want to be fast, and worse, the definition tries to pin down something that we expect to dispose of, in favor of a better replacement.

\section{Why is the innovation language paradox useful?}
\label{c3:s2}
The paradox is the engine of the approach for analysis and design of innovation languages. It is the temporary resolution of the paradox that moves us forward through cleaning up the content of innovation. 

The resolution is this: you want to define, in order to be precise, clear, and accurate in communication with others; in turn, this lets them be accurate, clear, and precise in what they disagree with, in what they want to change in what you provided; this puts them in your position, of putting forward their counterproposal, which itself needs to be accurate, clear, and precise, for you to improve on. 

This focus on definition helps reduce waste in communication, which is due to, roughly speaking, missing the point, digressing, and diverging in accidental directions. Moreover, it slows you down when you are preparing your communication about your ideas to others. It forces you to be accurate, clear, precise on your own first, before you ask for others' attention and time. Definitions expose your choices and preferences. And if you cannot be precise, accurate, and clear, then you simply do not know that which you are trying to define - this is far from a failure; it is, instead, a signal that you need others' help. If done well, the definitions you make set the standard for others, which eventually benefits to you.

It is convenient to think about the content of innovation as of an interconnected set of definitions. The notion of "a definition" is a common one. It is hard to meet someone who has never had to learn one. 

Even if these definitions we want to make are accurate, clear, and precise, they cannot be stable - after all, they are created during innovation, a process through which we learn about the problems and opportunities as we move towards a solution. 

Definitions that we make in this approach must be open to change, as few will be stable throughout the innovation process. Many will be thrown away. Some will make it, in a changed form; they will lose some pieces, get new ones, keeping a few all along. 

This is why we need a different kind of definition, than those we are used to.


\section{How to deal with the innovation language paradox?}
\label{c3:s3}
The paradox is addressed by making the innovation language from a special kind of definitions. 

A plastic definition is made to be shaped and reshaped easily, when and how we need it. Just like plastic objects, we want them to be easy and cheap to make, and to stay around for a long time if they have a use.

The innovation language paradox is about speed through precision, accuracy, and clarity, which slow down. Living with the paradox means that you need to define your ideas in such a way, that it invites change and highlights where these changes should likely occur. As we will see in more depth in the rest of the book, a plastic definition advertises its targets for improvement, showcases its limitations, and begs destruction. You make it in such a way that it is easier for others to find targets for improvement, refinement, or replacement. This makes it harder for you, not only because it is more work to do, but also because it requires you to make your choices transparent - and many of those will be refined and replaced, because others have something more convincing to offer.

What about relevance? How does a plastic definition relate to relevance of the content in an innovation process?

This is about how a definition begs action. You want to have definitions which incite team members to challenge them, which is why precision, accuracy, and clarity matter. I'll use blunt examples to illustrate this.

Let's say you and I need to design a new airplane. I'm looking for a design that is very different from anything already out there. At some point, I'm telling you that it will have such different wings, that anyone will know this is our design, as soon as they see them. But since there are innumerable wing designs (most not feasible, since I have no expertise relevant for this task), I might give you the following definition of these wings I will be calling x-wings:

\begin{svgraybox}
(Definition 1) 
X-wings: Oddly shaped wings for the new airplane design.
\end{svgraybox}

This is pretty much a useless definition, as far as the approach here is concerned. To see why, consider the following one instead.

\begin{svgraybox}
(Definition 2) 
X-wings: Four wings, two on each side of the airplane, the first pair placed 1/5th (or thereabout) of airplane's length from its tip, and the second pair placed 1/6th (open to change) of airplane's length from its tail, each with a width equal to 70\% of the airplane's length. The first pair is inclined at a 2 (might be more) degree angle up from the plane's horizontal axis, and the second is at a 2 degrees (more, but not above 4) angle towards the ground.
\end{svgraybox}

Why is the second definition any better? Because it is clearer, more precise, and - as these are the wings I was thinking of - it is more accurate to what I had in mind. 

It is important to know that I can, of course, be completely wrong about this wing design. It may not be feasible at all, or if it is, it may be impractical for many reasons. Can such an airplane be docked on a standard airport? How fast can it fly? How far? And so on. 

This is an important point: I could be proposing the wrong design, a crazy useless one, but the only way for you to tell me that is if I am clear, precise, and accurate about it. The reason for this should be clear: If I were to give you Definition 1, yet I had in mind what I described in Definition 2, then you would need to invest effort in discovering what I had in mind. I would expect you to ask me enough questions, that you can go from Definition 1 to Definition 2. This is hard to do, precisely because my thoughts are only mine, and yours are only yours, as I recalled in Chapter 1. 

There are two other major problems with me giving you Definition 1 when I could have given Definition 2. One, there is nothing that guarantees you will ask the questions you need to ask, to go from the former to the latter Definition. Two, isn't asking you to do it a waste of everyone's time? I have to wait for you to guess and ask the right questions, and answer your questions, while you struggle to find your way around something that may be so out of the ordinary (that wing design certainly is), that simply producing that definition proves a major challenge, only so that after all that is done, we can agree that design makes no sense at all. 

Five actions apply to plastic definitions: keep, refine, add, remove, and choose. Keep or refine what you agree with in a definition, add something that's missing, remove what you disagree with, and chose on my behalf for choices that I left open in the definition. 

To have something to agree or disagree with, the definition needs to expose your choices and options. It needs to make explicit, clear, precise, accurate (with regards to your ideas) statements, and the clearer, more precise, or accurate they are, I argue, the easier it is for you to take either of these actions.

This approach is hard work, and it is not a substitute for everything  that needs to be done in an innovation process anyway. There still needs to be research, brainstorming, prototyping, testing, and so on. But the aim with adding an Innovation  language is to have everyone be precise, clear, and accurate in communication, to avoid, or more realistically reduce misunderstandings, to ask everyone involved to be as explicit as possible, even if mistaken, about the ideas they are trying to convey, and to be transparent, when these ideas are yours, on what these ideas are, in ways which make it easier for others to agree, disagree, and build on what you are suggesting.

To close this chapter, remember two key ideas. One, the approach I am suggesting is for the team to make, improve, and maintain an innovation language. Two, the engine of this approach is the unstable resolution of the innovation language paradox: the pieces of an innovation language are precise, clear, and accurate, but malleable and disposable definitions; we invest in making them, not because we want to keep them, but to make it easier for everyone to replace, add to, or remove from them. The paradox is that we want precision, clarity, and accuracy from something we know is going to be thrown away or changed, and we see progress through that change, rather than stability. It's resolution is unstable because we go from one version of a definition to the next, and each time, if we are precise, clear, and accurate, it is easier to spot flaws, deficiencies, and incompleteness. 


% Chapter bibliography
\printbibliography